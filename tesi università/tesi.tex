\documentclass[a4paper,titlepage]{report}
\usepackage[T1]{fontenc} % codifica dei font
\usepackage[utf8]{inputenc} % lettere accentate da tastiera
\usepackage[italian]{babel} % lingua del documento
\usepackage{url} % per scrivere gli indirizzi Internet
\usepackage[sans,nouppercase]{frontespizio}

\begin{document}

\begin{frontespizio}

\Universita{Verona}
\Facolta{Informatica}
\Corso[Laurea]{Informatica}
\Titoletto{Tesi di laurea}
\Titolo{Internet of Things: un progetto di smart parking}
\Candidato{Aurora Bussola}
\Relatore{Prof. Nicola Fausto Spoto}
\Annoaccademico{2020-2021}
\Rientro{1.5cm}
\NCandidato{Laureando}
\Punteggiatura{}

\end{frontespizio}

\begin{abstract}
L'evoluzione digitale ha permesso ad Internet di espandersi coinvolgendo non solo computer networks ma pure diverse tipologie di dispositivi creando un legame fra il mondo virtuale e concreto. L'innovazione coinvolge un'infinità di ambiti, per questo motivo possiamo trovare dispositivi smart fra i più disparati: dalle macchinette del caffè connesse al WiFi di casa a dispositivi biomedicali come i pacemaker\cite{Ansari:BasedHealthcareApplications}. Tutto ciò trova ragione di esistere se si va ad analizzare la natura dell'uomo e del suo ambiente: siamo circondati da cose e più in particolare sopravviviamo grazie a delle cose ed è questo motivo per cui quest'ultime dovrebbero essere più importanti di qualsiasi idea e/o informazione\cite{Ashton: ThatInternetofThings'Thing}.\\La tesi, dopo una breve parte introduttiva, prenderà in esame un parcheggio intelligente per automobili trovando una possibile soluzione con l'ausilio di una piattaforma hardware per la prototipazione che sarà constantemente connessa ad Internet. Verranno descritti in particolare tutti gli aspetti e le scelte progettuali anche attraverso schemi concettuali, immagini e pseudocodici.
\end{abstract}
\tableofcontents
\chapter{Internet of Things}
\section{Definizione}
L'{\itshape Internet of Things}  abbrev. {\itshape IoT} (oppure {\itshape Internet delle Cose}, in italiano) può essere visto come un'estensione dell'Internet da noi conosciuto che aggiunge diverse tipologie di network e sensori. Esso incrementa l'ubiquità di Internet permettendo l'interconnessione di svariate tipologie di oggetti opportunamente configurati\cite{Feng:InternationJournalOfCommunicationSystems}.
\section{Una breve storia}
Il termine {\itshape Internet of Things} è stato coniato nel 1999 da Kevin Ashton, direttore dell'Auto-ID Centre\footnote{un'organizzazione no profit con sede presso il MIT} all'interno del quale venne inventata la tecnologia itshape RFID (acronimo di {\itshape Radio-Frequency IDentification}). L'invenzione soppianta il vecchio sistema barcode semplificando la gestione di beni in svariati ambiti: nel 2000 LG pianifica lo sviluppo un frigorifero intelligente in grado di {\itshape capire} se un determinato prodotto fosse presente o meno; tre anni dopo la tecnologia RFID entra nel programma SAVI\footnote{acronimo di {\itshape Navy sexual Assault Victim Interventition}, per la difesa alle vittime di tale delitto} e all'interno della grande catena di supermercati Walmart; nel 2008 viene esteso l'uso del protocollo IP a network di oggetti e nel 2011 nasce la {\itshape IPSO Alliance}, un forum di portata globale che arruola diversi colossi industriali per lo sviluppo dell’{\itshape Internet of Things}\cite{Suresh:StateArtReviewIoTHistoryTechnologyFieldsDeployment}.

\chapter{La tecnologia RFID}
% Bibliografia
\begin{thebibliography}{99}
\bibitem{Ansari:BasedHealthcareApplications}
Seema Ansari, Tahniyat Aslam, Javier Poncela, Pablo Otero, Adeel Ansari,
\emph{introduzione di Internet of Things-Based Healthcare Applications},
\url{https://www.igi-global.com/chapter/internet-of-things-based-healthcare-applications/243907}.
\bibitem{Ashton: ThatInternetofThings'Thing}
Kevin Ashton,
\emph{That 'Internet of Things' Thing},
\url{http://www.itrco.jp/libraries/RFIDjournal-That%20Internet%20of%20Things%20Thing.pdf}.
\bibitem{Feng:InternationJournalOfCommunicationSystems}
Feng Xia, Laurence T. Yang, Lizhe Wang, Alexey Vinel,
\emph{Internation Journal Of Communication Systems}
\bibitem{Suresh:StateArtReviewIoTHistoryTechnologyFieldsDeployment}
P. Suresh, J. Vijay Daniel, V. Parthasarathy, R. H. Aswathy,
\emph{A state of the art review on the Internet of Things (IoT) history, technology and fields of deployment}
\url{https://ieeexplore.ieee.org/abstract/document/7043637}.
\end{thebibliography}

\end{document}

